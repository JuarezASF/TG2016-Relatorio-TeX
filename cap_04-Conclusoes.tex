\chapter{Conclusões}

\label{CapConclusoes}

Desenvolveu-se um sistema que permite transferir um conjunto de movimentos
faciais para um avatar computacional a partir de uma sequência de imagens
capturadas por um par de câmeras. O sistema realiza a tarefa utilizando
\textit{webcams} simples e não exige um ambiente controlado, marcadores no
usuário ou mesmo uso de iluminação especial. Para se ter uma ideia da robustez
dos resultados, o sistema foi testado com sucesso em ambientes como uma sala de
escritório e até mesmo a sala de estar de um dos autores.

Os movimentos independentemente transferidos incluem: sorrir, abrir a boca,
fechar os olhos e levantar as sobrancelhas. A técnica chave utilizada foi
conectar características de um conjunto de pontos obtidos de um processo de
rastreamento visual com os pesos de mistura da técnica de Mistura de Poses,
permitindo gerar expressões variadas para o avatar a partir da combinação de um
conjunto reduzido de expressões chave. 

Para o rastreamento visual, utilizou-se a biblioteca FaceAnalysis SDK para
capturar 66 pontos chaves do rosto humano em uma sequência de imagens. Já para
gerar os pesos de mistura, foi utilizada uma regra simples baseada na razão
entre distâncias de pontos nomeados. Com o objetivo de tornar o processo robusto
à distância do usuário ao par de câmeras, se utilizou estimação de
tridimensionalidade para tornar os pesos de mistura uma função das distâncias
reais entre os pontos e não das distâncias em pixeis das câmeras. Para isso, foi
preciso uma etapa prévia de calibração das câmeras utilizadas.

Além disso, com o intuito de suavizar os resultados ao eliminar ruídos, os pesos
de mistura passam por um filtro digital. Dois tipos de filtros foram utilizados:
um filtro de de média móvel de Hanning de comprimento três e um conjunto de
filtros projetados pela técnica de corte de janela.  Apesar de muito menos
elaborado, o primeiro filtro apresentou melhores resultados que os últimos.
Entende-se que o ruído presente no sinal não possui comportamento espectral que
justifique um projeto cuidadoso de um filtro passa-baixas. Um filtro curto, que
ainda assim atenue frequências altas, performa melhor por impor um atraso menor
no sinal de saída.

Uma série de experimentos foi realizada com o objetivo de medir a confiança
em cada etapa do processo. A partir desses experimentos pôde-se inferir a
distância que o usuário deve se posicionar do par de câmeras para que a técnica
transfira os movimentos para o avatar de forma mais efetiva. Outro resultado foi
que apesar de adequada para capturar o movimento da boca e das sobrancelhas, a
regra associativa entre pontos e pesos de mistura não foi capaz de capturar
outros movimentos faciais, como o movimento de bochechas e o piscar dos olhos.
Isso se deve, em parte, à certa rigidez do algoritmo de rastreamento utilizado
quanto ao formato do rosto, o que impede que este seja aplicado, pelo menos
diretamente, para capturar nuâncias de certas expressões. Com isso, algumas das
poses chave disponíveis não foram utilizadas na aplicação final.

Os resultados obtidos mostram a validade da técnica. Por exemplo, o programa
desenvolvido foi capaz de sincronizar a abertura e o fechamento da boca do
avatar enquanto o usuário conversava com a câmera em um dos vídeos de teste.
Além disso, quando executando em máquinas comuns e ligado diretamente nas
câmeras, o programa mostra os resultados em tempo de execução sem atraso
perceptível entre entrada e saída. Por rodar em máquinas ordinárias, não exigir
longas horas de processamento e não exigir controle especial do ambiente de
filmagem a metodologia proposta se confirma como uma de baixo custo.



\section{Trabalhos Futuros}

A técnica de mistura de poses chaves se mostrou uma maneira simples de gerar
expressões complexas a partir de medidas simples obtidas em uma imagem. No
entanto, na implementação atual os pontos na malha das poses chaves estão
limitados a serem os mesmos dos pontos da pose base a menos de uma translação.
Com isso não pode-se animar uma pose chave onde, por exemplo, o avatar vira o
pescoço. A geometria vetorial na mistura de poses não funcionaria adequadamente.
É possível aplicar a técnica de mistura de poses com rotação dos pontos do
modelo, desde que se conheça a transformação completa que deve ocorrer em cada
ponto, não apenas sua translação. Uma possível direção para o trabalho seria
implementar estas transformações e as misturas delas. Nota-se que isso
implicaria, possivelmente, em mudar a representação dos modelos e acarretaria
maior custo na renderização. 

Ainda sobre as poses chaves, durante o projeto o artista gráfico foi capaz de
produzir mais poses bases do que criou-se regras associativas entre os pontos do
rastreamento visual e os pesos de mistura. Por exemplo, o programa não possui
regra capaz de reconhecer uma face com bochecha inchada. O problema principal é
que o algoritmo de rastreamento visual não segue bem o contorno da bochecha
nessa situação em particular, possivelmente devido à falta desta expressão nos
dados onde foi treinado. Uma direção natural para a continuação do trabalho é o
desenvolvimento de mais regras de associação entre imagem e pesos de mistura. O
processo de treinamento descrito em \cite{facetracker} poderia ser repetido,
sendo fornecido, dessa vez, um conjunto maior de dados de treinamento, incluindo
as expressões que se deseja capturar. Uma abordagem mais simples seria construir
rastreadores especializados que trabalhariam com dicas dos dados já capturados
pela FaceAnalysis SDK. Apesar do algoritmo de rastreamento desta SDK não ser
perfeito, ele fornece uma excelente pista de onde localizar as partes do rosto e
essa informação poderia ser utilizada para construir rastreadores específicos
para o que se quer medir. Por exemplo, é possível utilizar os pontos fornecidos
para delimitar uma pequena janela de busca ao redor da bochecha onde se
aplicaria algum truque para detectar bochechas cheias.

Tais melhorias poderiam tornar a execução do programa mais pesada e justificaria
o desenvolvimento de um código que melhor aproveitasse ferramentas de aceleração
gráfica. O código atual utiliza aceleração gráfica somente para renderizar o
resultado final, de forma que todas as outras operações são realizadas por
código  de máquina executado no processador principal  e em apenas uma linha de
execução. Outra justificativa para o uso de aceleração gráfica em outras etapas
da aplicação seria permitir que o programa rodasse suavemente com malhas mais
pesadas. Na implementação atual deste trabalho obtém-se atraso significativo
quando os modelos envolvidos apresentam dezenas de milhares de pontos.

Finalmente, em relação à captura de dados, a aplicação desenvolvida neste
trabalho utiliza apenas um par de câmeras baratas. Apesar de resultados
interessantes terem sido atingidos, certamente há limitações, ou no mínimo
dificuldades, impostas pelo método de captura utilizado.  Uma direção
interessante em que se poderia evoluir seria a adição de um sensor de nuvem de
pontos, como o Kinect, ao pipeline das técnicas. Apesar deste tipo de sensor ser
mais caro que um par de câmeras, a adição de um sensor de nuvem de pontos não
desclassificaria a aplicação como uma de baixo custo, ao se considerar o custo
total do desenvolvimento de animação. Porém, como essa adição mudaria
drasticamente o formato dos dados, seria necessário também desenvolver novas
regras que associem pesos de mistura à nuvem de pontos.

