\chapter{Resultados}

Este capítulo apresenta os resultados obtidos nos experimentos feitos para
validação das técnicas utilizadas neste trabalho, bem como o resultado final da
aplicação. Buscou-se repetir os experimentos várias vezes (mais do que 30 vezes)
a fim de garantir valor estatístico para os resultados apresentados.

\section{Calibração das Câmeras}

A Tabela \ref{tab:params_intrisecos} mostra os parâmetros intrínsecos das
câmeras utilizadas, obtidos a partir da ferramenta \textit{Camera Calibration
Toolbox for Matlab}.

\begin{table}[!htb]
\centering
\begin{tabular}{|c|c|c|}
\hline
Parâmetros intrínsecos & Câmera 1 (pixel) & Câmera 2 (pixel)\\ \hline
$f_c(1)$ & $854 \pm 4$ &  $834 \pm 6$ \\ \hline
$f_c(2)$ & $858 \pm 4$ & $836 \pm 6$ \\ \hline
$cc(1)$ & $298 \pm 6$ & $300 \pm 7$ \\ \hline
$cc(2)$ & $232 \pm 6$ & $217 \pm 7$ \\ \hline
$alpha_c$ & 0 & 0 \\ \hline
\end{tabular}

\caption{Parâmetros intrínsecos da câmera medidos em pixeis.}
\label{tab:params_intrisecos}
\end{table}

Como pode ser observado os valores de $f_c(1)$ e $f_c(2)$, referentes a
distância focal medida em pixeis horizontais e verticais, são bem próximos para
ambas as câmeras. Essa verificação valida a aproximação assumida, durante a
modelagem, de que câmera possui um pixel quadrado. Portanto, a distância focal
utilizada, para efeitos de cálculo, é dada por $f = (f_c(1) + f_c(2))/2$. Os
valores são dados em pixeis.

Os resultados similares obtidos para cada uma das câmeras não é ao acaso, visto
que as duas câmeras utilizadas são da mesma marca e modelo. Além disso, ambas
possuem foco ajustável, o ajuste em cada uma foi feito manualmente e por
avaliação visual, de modo a capturar imagens nítidas.

Os parâmetros calculados fornecem os elementos necessários para tornar possível
a estimação da profundidade de um ponto alvo, a partir da Equação
\ref{eq:3d_Zequation}. Assim, pode-se estimar o valor de $X$ em centímetros,
pelas Equações \ref{eq:3d_realX1} e \ref{eq:3d_realX1}, bem como estimar o valor
de $Y$ em centímetros, pelas Equações \ref{eq:3d_realY1} e \ref{eq:3d_realY2}.

\section{Rastreamento de Pontos da Face}

Os experimentos realizados nesta seção foram feitos com o objetivo de avaliar
isoladamente a precisão da detecção de pontos da face utilizando a SDK
\textit{CSIRO Face Analysis}, ou seja, as medidas são tomadas com os valores
puros (em pixeis e não filtrados) obtidos pelo rastreamento. O objetivo desta
análise é permitir a comparação com os resultados obtidos após filtragem e
estimação de tridimensionalidade, bem como justificar o uso destas técnicas.


Escolheu-se medir a razão de distância para o \textbf{comprimento dos lábios} em
repetidas tomadas, para duas das poses disponíveis: a Pose Neutra e a Pose
Sorrindo. Essa razão de distância, após filtragem, compõe o peso de mistura para
a Pose Sorrindo na aplicação final e foi escolhida por ser a razão que apresenta
maior amplitude de variação entre os extremos máximos e mínimos.

Para ambas câmeras 1 e 2 obteve-se a média e o desvio padrão, em pixeis, para
essa razão de distância. Esses dados foram adquiridos a medida que o usuário se
posicionava a diferentes distâncias em relação à montagem de captura, variando,
assim, os valores de profundidade. Os dados de média e desvio padrão foram
obtidos a partir de medições em trinta imagens de uma face imóvel, capturadas
sucessivamente com intervalos de tempo menores que um segundo.

A Tabela \ref{tab:exp-px-variando-dist-cam1} exibe os valores obtidos através da
câmera 1 enquanto a Figura \ref{fig:graf-cam1-dsv-dist} mostra um gráfico da
variação do desvio padrão do comprimento dos lábios, para a Pose Neutra e a para
Sorrindo, em relação a distância entre a câmera 1 e o alvo.


\begin{table}[!htb]
\centering
\begin{tabular}{|*{6}{>{\centering\arraybackslash}p{.16\linewidth}|}}
\cline{2-5}
	\multicolumn{1}{c|}{} & \multicolumn{2}{c|}{Câmera 1 - Pose Neutra (pixel)} & \multicolumn{2}{c|}{Câmera 1 - Pose Sorrindo (pixel)} \\ \hline
    \begin{tabular}{@{}c@{}}Distância da \\ câmera (cm)\end{tabular} & Média &  Desvio padrão & Média &  Desvio padrão \\ \hline
40 & 93.4096 & 0.5575 & 92.1488 & 1.3473 \\ \hline
45 & 83.7242 & 0.6406 & 84.694  & 0.8848 \\ \hline
50 & 75.6548 & 0.7163 & 78.1044 & 1.0881 \\ \hline
55 & 67.9294 & 0.6772 & 69.1071 & 0.9671 \\ \hline
60 & 65.1105 & 0.3884 & 65.8582 & 0.8114 \\ \hline
65 & 58.8054 & 0.3392 & 59.7229 & 0.5219 \\ \hline
70 & 54.7137 & 0.5925 & 55.7597 & 0.6367 \\ \hline
75 & 53.4555 & 0.4781 & 53.7684 & 0.6277 \\ \hline
80 & 49.1185 & 0.3156 & 49.7219 & 0.2373 \\ \hline
\end{tabular}

\caption{Comprimento horizontal dos lábios medido em pixeis a partir de imagens
capturadas do rosto alvo pela câmera 1 em distâncias variáveis.}

\label{tab:exp-px-variando-dist-cam1}
\end{table}

\begin{figure}[!htb]
\centering
\includegraphics[width=0.8\textwidth]{figs/thumbnail_cameraEsquerda.jpg} 

\caption{Gráfico da variação do desvio padrão, em pixeis, em relação a
distância, em centímetros, de pontos rastreados nas imagens capturadas pela
câmera 1.}

\label{fig:graf-cam1-dsv-dist}
\end{figure}

A Tabela \ref{tab:exp-px-variando-dist-cam2} exibe os valores obtidos através da
câmera 2 e a Figura \ref{fig:graf-cam2-dsv-dist} mostra um gráfico da variação
do desvio padrão do comprimento dos lábios, para a Pose Neutra e a para
Sorrindo, em relação a distância entre a câmera 2 e o alvo.

\begin{table}[!htb]
\centering
\begin{tabular}{|*{6}{>{\centering\arraybackslash}p{.16\linewidth}|}}
\cline{2-5}
	\multicolumn{1}{c|}{} & \multicolumn{2}{c|}{Câmera 2 - Pose Neutra (pixel)} & \multicolumn{2}{c|}{Câmera 2 - Pose Sorrindo (pixel)} \\ \hline
    \begin{tabular}{@{}c@{}}Distância da \\ câmera (cm)\end{tabular} & Média &  Desvio padrão & Média &  Desvio padrão \\ \hline
40 & 92.1488 & 1.3473 & 119.6076 & 2.2723 \\ \hline
45 & 84.694  & 0.8848 & 106.7452 & 2.23   \\ \hline
50 & 78.1044 & 1.0881 & 108.4763 & 1.2881 \\ \hline
55 & 69.1071 & 0.9671 & 94.7001  & 1.2003 \\ \hline
60 & 65.8582 & 0.8114 & 87.5216  & 0.5491 \\ \hline
65 & 59.7229 & 0.5219 & 82.719   & 0.9558 \\ \hline
70 & 55.7597 & 0.6367 & 75.3994  & 0.4099 \\ \hline
75 & 53.7684 & 0.6277 & 67.9172  & 1.0836 \\ \hline
80 & 49.7219 & 0.2373 & 68.2629  & 1.0946 \\ \hline
\end{tabular}

\caption{Comprimento horizontal dos lábios medido em pixeis a partir de imagens
capturadas do rosto alvo pela câmera 2 em distâncias variáveis.}

\label{tab:exp-px-variando-dist-cam2}
\end{table}

\begin{figure}[!htb]
\centering
\includegraphics[width=0.8\textwidth]{figs/thumbnail_cameraDireita.jpg} 

\caption{Gráfico da variação do desvio padrão, em pixeis, em relação a
distância, em centímetros, de pontos rastreados nas imagens capturadas pela
câmera 2.}

\label{fig:graf-cam2-dsv-dist}
\end{figure}

O desvio padrão é importante pois funciona como um indicador da precisão do
processo de medida. Fosse o processo regido por uma distribuição puramente
Gaussiana, o desvio padrão do processo indicaria que aproximadamente 66\% das
medidas serão observadas dentro de uma distância igual ao desvio padrão da média
do processo, que é aceito como o valor mais provável para o valor verdadeiro
sendo medido.  Os gráficos exibidos na Figura \ref{fig:graf-cam1-dsv-dist} e na
Figura \ref{fig:graf-cam2-dsv-dist} mostram que existe incerteza no rastreamento
de pontos da face e que a incerteza apresenta um comportamento que varia com a
distância. Apesar de o desvio padrão ter sido pequeno quando comparado às
dimensões de cada uma das das imagens capturadas, a existência dessa imprecisão
pode afetar a qualidade da aplicação final. Além disso, outra análise
interessante é que, como as imagens onde o rastreamento foi aplicado foram
tomadas em intervalos de tempo muito curtos, e, ainda assim, houve variação nas
medidas realizadas, diz-se que essas variações constituem um sinal de erro que
deve possuir componentes de elevada frequência. Isto é, o ruído presente no
processo de medida deve possuir componentes rápidas o suficiente para que seja
capaz de influenciar medidas tomadas em instantes de tempo muito próximos uns
dos outros. Essa observação motiva o uso de filtragem digital com o objetivo de
cortar tais altas frequências.

Outro comportamento exibido pelos gráficos é o de que o desvio padrão é quase
sempre maior quando a razão de distância é medida na Pose Sorrindo, do que
quando ela é medida na Pose Neutra, ou seja, o rastreamento geralmente é mais
preciso quando tomado em uma face sem expressões.  Este resultado pode ser um
reflexo do erro de truncamento introduzido pela escolha de apenas algumas das
configurações encontradas pela análise PCA e que, aparentemente, as componentes
escolhidas privilegiam configurações neutras.

Outra observação que pode ser tirada é que o desvio padrão em geral diminui com
o aumento da distância. Esse acontecimento pode ser atribuído ao fato de que,
como neste experimento as medidas são avaliadas em pixeis, quanto maior a
distância do usuário a câmera menor o valor da razão de distância.

Finalmente, observando as tabelas é possível reparar a diminuição drástica da
média das medidas conforme o valor de profundidade aumenta: vê-se que o valor
obtido para o comprimento dos lábios a uma distância de 80cm é quase a metade do
valor obtido a 40cm.  Essa mudança brusca impossibilita uma boa determinação das
distâncias mínimas e máximas, necessárias para a boa calibração de uma razão de
distância, uma vez que estes parâmetros mudam com uma variação de profundidade,
mesmo se o alvo estiver na mesma pose.

Estas duas últimas observações motivam o uso de uma técnica de medida que
produza valores invariantes a distância que o usuário se encontra do sistema de
captura. Esta técnica deve produzir sempre uma mesma razão de distância para uma
mesma pose, independentemente da proximidade do usuário nas imagens de entrada.

\section{Estimação de Tridimensionalidade	}

O objetivo dos experimentos nessa seção é avaliar a precisão da técnica de
estimação tridimensional. Deve-se notar que uma precisão na estimação do $X$ e
do $Y$ no sistema de coordenadas do mundo só é possível caso haja precisão na
estimação do $Z$, como pode ser observado nas Equações \ref{eq:3d_realX1},
\ref{eq:3d_realX2}, \ref{eq:3d_realY1} e \ref{eq:3d_realY2}.

Como nos experimentos da seção anterior, a razão de distância analisada foi a
referente ao comprimento dos lábios. Foram utilizadas as mesmas imagens
capturadas no experimento anterior, porém desta vez as medidas foram tomadas
após o par de imagens ter sido combinado com os parâmetros intrínsecos da câmera
para produzir medidas em centímetros, ou seja, os pontos foram estimados no
sistema de coordenadas do mundo.

A Tabela \ref{tab:exp-px-variando-dist-3d} exibe os valores estimados em
centímetros para a média e o desvio padrão do comprimento dos lábios para as
poses Neutra e Sorrindo. Já a Figura \ref{fig:graf-3d-dsv-dist} e a Figura
\ref{fig:graf-3d-media-dist} mostram gráficos para estes valores a medida que o
usuário se afasta do sistema de captura.


\begin{table}[!htb]
\centering
\begin{tabular}{|*{6}{>{\centering\arraybackslash}p{.16\linewidth}|}}
\cline{2-5}
	\multicolumn{1}{c|}{} & \multicolumn{2}{c|}{Pose Neutra (cm)} & \multicolumn{2}{c|}{ Pose Sorrindo (cm)} \\ \hline
    \begin{tabular}{@{}c@{}}Distância da \\ câmera (cm)\end{tabular} & Média &  Desvio padrão & Média &  Desvio padrão \\ \hline
40 & 5.2755 & 0.1155 & 6.8652 & 0.2549 \\ \hline
45 & 5.5857 & 0.0991 & 6.8311 & 0.1564 \\ \hline
50 & 5.4356 & 0.0959 & 7.3474 & 0.116  \\ \hline
55 & 5.3663 & 0.1135 & 7.3599 & 0.1574 \\ \hline
60 & 5.5722 & 0.1144 & 7.4396 & 0.1107 \\ \hline
65 & 5.4381 & 0.0932 & 7.4996 & 0.1433 \\ \hline
70 & 5.8243 & 0.0879 & 7.7648 & 0.1024 \\ \hline
75 & 5.7759 & 0.1494 & 7.1711 & 0.2339 \\ \hline
80 & 5.5645 & 0.0716 & 7.7803 & 0.1659 \\ \hline
\end{tabular}

\caption{Comprimento horizontal dos lábios medido em centímetros a partir de
imagens capturadas do rosto alvo em distâncias variáveis.}

\label{tab:exp-px-variando-dist-3d}
\end{table}

\begin{figure}[!htpb]
\centering
\includegraphics[width=0.8\textwidth]{figs/thumbnail_distanciaCorrigida.jpg} 

\caption{Gráfico da variação do desvio padrão, em centímetros, em relação a
distância, em centímetros, de pontos estimados no sistema de coordenadas do
mundo.}

\label{fig:graf-3d-dsv-dist}
\end{figure}

O gráfico exibido na Figura \ref{fig:graf-3d-dsv-dist} mostra como se comporta o
desvio padrão a medida que o usuário se distancia do sistema de captura. O
comportamento é similar ao observado nas Figuras \ref{fig:graf-cam1-dsv-dist} e
\ref{fig:graf-cam2-dsv-dist}, revelando a característica estocástica do processo
de medida adotado. A diferença entre as Figuras  \ref{fig:graf-3d-dsv-dist},
\ref{fig:graf-cam1-dsv-dist} e \ref{fig:graf-cam2-dsv-dist} é que a primeira
apresenta os resultado quando as medidas são tomadas no sistema de coordenadas
estimadas do mundo enquanto as últimas apresentam resultados quando medidas são
tomadas diretamente nos planos das câmeras. Vê-se que a estimação de
profundidade não altera este comportamento do sistema de medida.

\begin{figure}[!htpb]
\centering
\includegraphics[width=0.8\textwidth]{figs/media3d.png} 

\caption{Gráfico da variação média, em centímetros, em relação a distância, em
centímetros, de pontos estimados no sistema de coordenadas do mundo.}

\label{fig:graf-3d-media-dist}
\end{figure}

A precisão da estimação tridimensional pode ser verificada ao se avaliar a
variação da média em relação a distância. Esses dados são exibidos na Figura
\ref{fig:graf-3d-media-dist}, nota-se que a média se mantém aproximadamente
constante, principalmente nos intervalos entre 50cm e 65cm. Com isso pode-se
concluir que mesmo que aconteça uma variação da distância do alvo em relação as
câmeras, a razão de distância se manterá estável, garantindo estabilidade para o
modelo final.

\section{Filtros}

Nestes experimentos foram gerados gráficos para as sequências dos pesos de
mistura com e sem filtragem de três das poses utilizadas. Cada gráfico mostra
medidas de pesos de mistura para uma das poses, com um curva para os valor
não-filtrado e curvas para saídas de alguns dos filtros projetados.  As
sequências foram obtidas em um vídeo gravado com o objetivo de movimentar
especificamente a pose sendo testada. Os gráficos foram gerados selecionando-se
manualmente parte das sequências gravadas que produziram comportamento
interessante para comparação.

A Figuras \ref{fig:filter-left-eye}, \ref{fig:filter-open-mouth} e
\ref{fig:filter-smile} mostram a atuação dos filtros sobre a variação do peso de
mistura para as poses Olho Esquerdo, Boca Aberta e Sorriso respectivamente.
Nestes gráficos, mostra-se sempre a saída não filtrada, a saída filtrada com
filtro de média móvel de Hanning e a saída filtrada com filtro de projeto com
janela de Hamming para algumas frequências de corte. Os filtros com projeto em
janela tem todos comprimento 16.

\begin{figure}[!htb]
\centering
\includegraphics[width=1.0\textwidth]{figs/filter-result-open-mouth.pdf} 
\caption{Peso de mistura para a Pose Olho Esquerdo}
\label{fig:filter-left-eye}
\end{figure}

\begin{figure}[!htb]
\centering
\includegraphics[width=1.0\textwidth]{figs/filter-result-left-eye.pdf} 
\caption{Peso de mistura para a Pose Boca Aberta}
\label{fig:filter-open-mouth}
\end{figure}

\begin{figure}[!htb]
\centering
\includegraphics[width=1.0\textwidth]{figs/filter-result-smile.pdf} 
\caption{Peso de mistura para a Pose Sorriso}
\label{fig:filter-smile}
\end{figure}

Como visto nas Figuras \ref{fig:graf-cam1-dsv-dist} e
\ref{fig:graf-cam2-dsv-dist} dos experimentos anteriores, o rastreamento em si
apresenta instabilidades mesmo quando o rosto está imóvel. Portanto, pode-se
esperar que em um vídeo longo onde ocorre movimentação e variação de
luminosidade o sinal sem filtragem seja ainda mais instável. Esse comportamento
é visto nas oscilações rápidas observadas nos gráficos em
\ref{fig:filter-left-eye}, \ref{fig:filter-open-mouth} e \ref{fig:filter-smile}. 

A filtragem suaviza o sinal e consequentemente remove oscilações indesejáveis na
animação final. Vários tipos de filtro foram analisados e seus desempenhos foram
variados. Como pode ser observado, os filtros projetados pela técnica de janela
foram compridos de mais e introduzem atrasos no sinal de saída em relação ao de
entrada sinal e, apesar de eles serem mais poderosos em filtrar altas
frequências, esse tipo de filtro acaba prejudicando a performance da animação em
tempo real. 

Por outro lado o filtro de média móvel de Hanning segue adequadamente o sinal
original e mantem uma boa atenuação das altas frequências. Esse filtro introduz
um menor atraso em seguir o sinal de entrada, pois possui muito menos
coeficientes que os outros filtros projetados. Para este trabalho, onde um dos
objetivos é a animação em tempo real, o filtro de Hanning apresentou o melhor
resultado. Poderia ter sido experimentado um projeto em janela utilizando-se
menos coeficientes, mas como o filtro de média móvel mostrou bons resultados,
decidiu-se por não prosseguir com a experimentação de mais filtros.

\section{Mistura de Poses}

Neste experimento trabalhou-se apenas com a técnica de Mistura de Poses para
compor um modelo final sem a influência do rastreamento de pontos da face. Isto
foi feito para ser possível avaliar esta técnica isoladamente. Para isso foram
atribuídos manualmente os valores dos pesos de mistura.

Alguns exemplos de modelos finais mostrando poses intermediárias renderizadas
com sucesso podem ser vistos na Figura
\ref{fig:blend-shapes-inter-simple-shapes}.

\begin{figure}[!htb]
  \centering
  \begin{subfigure}[]{\label{fig:inter1}\includegraphics[width=0.4\textwidth]{./figs/TG_angry60_leftcheek75_lefteye55_closemouth35.png}}
  \end{subfigure}   
  \begin{subfigure}[]{\label{fig:inter2}\includegraphics[width=0.4\textwidth]{./figs/TG_happy40_righteyebrow95_openmouth60.png}}
  \end{subfigure}
  
  \begin{subfigure}[]
  {\label{fig:inter3}\includegraphics[width=0.4\textwidth]{./figs/TG_lefteye100_rigtheye100_openmouth60.png}}
  \end{subfigure} 
  \begin{subfigure}[]
  {\label{fig:inter4}\includegraphics[width=0.4\textwidth]{./figs/TG_happy70_angry90.png}}
  \end{subfigure}
  
  \begin{subfigure}[]{\label{fig:inter5}\includegraphics[width=0.4\textwidth]{./figs/TG_angry100_openmouth90.png}}
  \end{subfigure}
  \begin{subfigure}[]{\label{fig:inter6}\includegraphics[width=0.4\textwidth]{./figs/TG_angry100_leftcheek100_rightcheek65_closemouth100.png}}
  \end{subfigure}

  \caption{Exemplo de misturas geradas configurando os parâmetros de mistura
    manualmente. Pesos da Figura \ref{fig:inter1}: Olho Esquerdo - 0.55, Boca
    Fechada - 0.35, Bochecha Esquerda - 0.75 e Bravo - 0.6. Pesos da Figura
    \ref{fig:inter2}: Sobrancelha Direita - 0.95, Boca Aberta - 0.6 e Sorrindo -
    0.4. Pesos da Figura \ref{fig:inter3}: Olho Esquerdo - 1.0, Olho Direito -
    1.0 e Boca Aberta - 0.6. Pesos da Figura \ref{fig:inter4}: Sorrindo - 0.7 e
    Bravo - 0.6. Pesos da Figura \ref{fig:inter5}: Boca Aberta - 1.0 e Bravo -
    0.9. Pesos da Figura \ref{fig:inter6}: Boca Fechada - 1.0, Bochecha Esquerda
  - 1.0, Bochecha Direita - 0.65, e Bravo - 1.0.}


  \label{fig:blend-shapes-inter-simple-shapes}
\end{figure}

Como pode ser observado nos modelos exibidos, a técnica de mistura de poses teve
ótimos resultados. Os resultados mostram que a implementação utilizando
indexação de VBOs e que a ordenação dos vetores foi adequada. Não fosse esse o
caso, seria possível observar falhas no modelo, como  buracos ou presença de
deformações irregulares.

Pode-se notar que a presença de cada pose chave no modelo final está bem
relacionada com os pesos aplicados. Verifica-se que a aplicação da Equação
\ref{eq:blendshapes} permite criar, a partir de poucas poses pré-definidas, uma
quantidade enorme de poses intermediárias significativamente diferentes.

Com esses resultados nota-se que se o programa for capaz de gerar pesos de
mistura que adequadamente correspondam às expressões apresentadas pelo usuário,
a técnica de mistura de poses será capaz de compor um modelo final com
expressões variadas.

\section{Sistema em Funcionamento}

Para que o sistema funcione adequadamente é necessária uma etapa de calibração
das distâncias mínimas e máximas observadas em cada uma das razões de distância.
Isso é feito de forma manual ao executar o programa, pois toma-se nota de
valores impressos em tela sobre as distâncias entre os pontos tomadas no quadro
atual. Por exemplo, pode-se sorrir e, então, anotar o valor medido para o
sorriso que é impresso na tela. Os valores anotados são, posteriormente,
inseridos no código para serem utilizados para cálculo das razões de distância.
Os valores utilizados para a sequência de resultados obtidos a seguir são
apresentados na Tabela \ref{tab:distance-ratio-coeffs}.

\begin{table}
\centering
\begin{tabular}{|c|c|c|}
\hline
pose & distância mínima (cm) & distância mínima (cm) \\ \hline
sorriso & 5.9 & 7.9 \\ \hline
sobrancelha esquerda & 2.4 & 3.7 \\ \hline
sobrancelha direita & 2.4 & 3.6 \\ \hline
olho esquerdo & 0.3 & 0.6 \\ \hline
olho direito & 0.3 & 0.6 \\ \hline
boca aberta & 0.3 & 4.3 \\ \hline
\end{tabular}
\caption{Valores de distância mínima e distância máxima definidos, para cada pose, após calibração.}
\label{tab:distance-ratio-coeffs}
\end{table}

Utiliza-se os coeficientes acima para rodar a aplicação sobre vídeos de entrada
e guarda-se alguns quadros dos vídeos e da saída.  As Figuras
\ref{fig:sist-func}, \ref{fig:sist-func-2}, \ref{fig:sist-func-3} e
\ref{fig:sist-func-4} mostram lado a lado quadros de entrada e os resultados da
transferência de expressão para o avatar. Nas Figuras de \ref{fig:sist-func} a
\ref{fig:sist-func-3} os quadros de entrada foram escolhidos de forma a mostrar
expressões variadas que podem ser capturadas pela técnica proposta. A figura
\ref{fig:sist-func-4} mostra uma sequência de três quadros consecutivos tirada
enquanto o usuário conversava com a câmera. Notar que a boca do avatar abre
seguindo o movimento do usuário.

Os resultados mostram que o acoplamento entre o rastreamento de pontos do rosto
com os pesos de mistura por meio de razões de distâncias se deu de forma
satisfatória. Há poses para mistura que não foram utilizadas, mas todas as poses
para as quais se definiu uma razão de distância foram adequadamente transferidas
do usuário para o avatar. O avatar é capaz de sorrir, abrir a boca, movimentar
as sobrancelhas e piscar segundo o movimento do usuário. 


\begin{figure}[!htb]
  \centering
  \begin{subfigure}[]{\label{fig:inter1}\includegraphics[width=0.35\textwidth]{./figs/TG-resultado-par-4-img-1.png}}
  \end{subfigure}   
  \begin{subfigure}[]{\label{fig:inter2}\includegraphics[width=0.35\textwidth]{./figs/TG-resultado-par-4-img-2.png}}
  \end{subfigure}
  
  \begin{subfigure}[]
  {\label{fig:inter3}\includegraphics[width=0.35\textwidth]{./figs/TG-resultado-par-5-img-1.png}}
  \end{subfigure} 
  \begin{subfigure}[]
  {\label{fig:inter4}\includegraphics[width=0.35\textwidth]{./figs/TG-resultado-par-5-img-2.png}}
  \end{subfigure}
  
    \begin{subfigure}[]
  {\label{fig:inter3}\includegraphics[width=0.35\textwidth]{./figs/i_0230.png}}
  \end{subfigure} 
  \begin{subfigure}[]
  {\label{fig:inter4}\includegraphics[width=0.35\textwidth]{./figs/o_0250.png}}
  \end{subfigure}

  \caption{Imagens que demonstram o sistema em funcionamento.  Em relação ao
    movimento da boca, as Figuras \ref{fig:sist-func}(d) e
    \ref{fig:sist-func}(f) deixam claro que há dois movimentos sendo rastreados
  independentemente: abrir a boca verticalmente como no ato de bocejar e abri-la
horizontalmente como no ato de sorrir.}

  \label{fig:sist-func}
\end{figure}

\begin{figure}[!htb]
  \centering
  \begin{subfigure}[]{\label{fig:inter1}\includegraphics[width=0.35\textwidth]{./figs/i_0330.png}}
  \end{subfigure}   
  \begin{subfigure}[]{\label{fig:inter2}\includegraphics[width=0.35\textwidth]{./figs/o_0370.png}}
  \end{subfigure}
  
  \begin{subfigure}[]
  {\label{fig:inter3}\includegraphics[width=0.35\textwidth]{./figs/i_0350.png}}
  \end{subfigure} 
  \begin{subfigure}[]
  {\label{fig:inter4}\includegraphics[width=0.35\textwidth]{./figs/o_0380.png}}
  \end{subfigure}
  
    \begin{subfigure}[]
  {\label{fig:inter3}\includegraphics[width=0.35\textwidth]{./figs/i_0550.png}}
  \end{subfigure} 
  \begin{subfigure}[]
  {\label{fig:inter4}\includegraphics[width=0.35\textwidth]{./figs/o_0640.png}}
  \end{subfigure}
  
   \begin{subfigure}[]
  {\label{fig:inter3}\includegraphics[width=0.35\textwidth]{./figs/i_0780.png}}
  \end{subfigure} 
  \begin{subfigure}[]
  {\label{fig:inter4}\includegraphics[width=0.35\textwidth]{./figs/o_0850.png}}
  \end{subfigure}

  \caption{Imagens que demonstram o sistema em funcionamento. Compare a Figura
    \ref{fig:sist-func}(d) com a \ref{fig:sist-func-2}(d) para notar que o
  avatar está sorrindo com a boca fechada nesta e aberta naquela. } 

  \label{fig:sist-func-2}
\end{figure}

\begin{figure}[!htb]
  \centering
  \begin{subfigure}[]{\label{fig:inter1}\includegraphics[width=0.35\textwidth]{./figs/i_0400.png}}
  \end{subfigure}   
  \begin{subfigure}[]{\label{fig:inter2}\includegraphics[width=0.35\textwidth]{./figs/o_0490.png}}
  \end{subfigure}
  
  \begin{subfigure}[]
  {\label{fig:inter3}\includegraphics[width=0.35\textwidth]{./figs/i_0140.png}}
  \end{subfigure}
  \begin{subfigure}[]
  {\label{fig:inter4}\includegraphics[width=0.35\textwidth]{./figs/o_0160.png}}
  \end{subfigure}

  \caption{Imagens que demonstram o sistema em funcionamento. Novamente, o
  avatar pode sorrir com a boca fechada ou com a boca aberta, podendo o olho
estar fechado e aberto. As poses são combinadas independentemente.}

  \label{fig:sist-func-3}
\end{figure}

\begin{figure}[!htb]
  \centering
  \begin{subfigure}[]{\label{fig:inter1}\includegraphics[width=0.35\textwidth]{./figs/i_0250.png}}
  \end{subfigure}   
  \begin{subfigure}[]{\label{fig:inter2}\includegraphics[width=0.35\textwidth]{./figs/o_0320.png}} 
  \end{subfigure}  
  
  \begin{subfigure}[]
  {\label{fig:inter3}\includegraphics[width=0.35\textwidth]{./figs/i_0260.png}}
  \end{subfigure} 
  \begin{subfigure}[]
  {\label{fig:inter4}\includegraphics[width=0.35\textwidth]{./figs/o_0330.png}}
  \end{subfigure}
  
    \begin{subfigure}[]
  {\label{fig:inter4}\includegraphics[width=0.35\textwidth]{./figs/i_0270.png}}
  \end{subfigure}
    \begin{subfigure}[]
  {\label{fig:inter4}\includegraphics[width=0.35\textwidth]{./figs/o_0340.png}}
  \end{subfigure}

  \caption{Imagens que demonstram o sistema em funcionamento \textit{frame} a
  \textit{frame}. A sequência mostra a abertura gradual da boca enquanto o
personagem conversa com as câmeras.}

  \label{fig:sist-func-4}
\end{figure}




